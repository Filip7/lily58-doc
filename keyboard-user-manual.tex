\documentclass{article}

\title{Lily58 User Manual}
\author{Filip M.}
\date{\today}

\usepackage[square,numbers]{natbib}
\bibliographystyle{abbrvnat}
\usepackage{hyperref}

\begin{document}

\maketitle

\tableofcontents

\section{Introduction}

Thank you for buying Lily58 keyboard and build service from me!
I really enjoy using and making this keyboard, and I hope you will to!

\section{Configuration of the keyboard}

There are couple of ways to configure this keyboard. Best ways are using QMK directly and/or VIA software.
We will dive deep into each one.
But there are some other firmewares that I haven't used, so I cannot reccomend them based on not enough of experience with them.
\textbf{Remember we are deaing with flashing firmeware!} So thread carefully and do not fry your chip* :D
* chip is easily replaceable in case of breakage

\subsection{Quick Overview}
Here is quick overview and description of each tool
\begin{table}
	\caption{Configuration tools}\label{tab:}
	\begin{center}
		\begin{tabular}[c]{l|l|l}
			\hline
			\multicolumn{1}{c|}{\textbf{}} & 
			\multicolumn{1}{c}{\textbf{}} \\
			\hline
			QMK & link & desc \\
			VIA & link & desc \\
			ZMK & link & desc \\
			KMK & link & desc \\
			
			\hline
		\end{tabular}
	\end{center}
\end{table}

\subsection{QMK}

\subsection{VIA}

\subsection{ZMK}

\subsection{KMK}

\section{Hardware description}

\section{Be aware of}
% Make this a Q\\&A format??
Some oddities I found using the keyboard.
Pluging in and out the usb-c cable under some angle could pull the chip from the socket, so it may seem that the keyboard has stopped working, just gently push the chip back to the socket, ideally, do this when the keyboard is unplugged, you do not want to short circuit something :)

\bibliography{sources.bib}

\end{document}

